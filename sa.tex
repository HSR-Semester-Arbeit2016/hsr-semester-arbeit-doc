\documentclass[a4paper]{article}
%Usage
\usepackage{datetime} 
\usepackage{amsmath}
\usepackage{mathptmx}
\usepackage{hyperref}
\usepackage{graphicx, wrapfig}
\usepackage{caption}
\usepackage{subcaption}
\graphicspath{ {images/} }
\usepackage{verbatim}
\usepackage{geometry}
\usepackage{wrapfig}
\geometry{a4paper, top=1cm, bottom=1cm, margin=1.5cm , headheight=12.5pt}
\usepackage{booktabs}
\usepackage[latin1]{inputenc}
% u.a. Deutsche Sonderzeichen.
  % Je nach genutztem System wird ein anderer Encodierer ben�tigt:
  % \usepackage[Encodierer]{inputenc}
  % Encodierer		System
  % latin1, latin9	UNIX, Linux
  % ansinew		Windows
  % applemac		Apple
\usepackage{pxfonts}
\usepackage{amssymb}

\hypersetup{
    pdftitle={Semester Arbeit},    % title
    pdfauthor={Roberto Cuervo Alvarez},     % author
    pdfauthor={Konrad H�pli},     % author
    pdfsubject={Augmented Reality Android App},   % subject of the document
    pdfcreator={pdftex}  % creator of the document
}
%Header and footer
\usepackage{fancyhdr}
\pagestyle{fancy}
\fancyhead[RO,LE]{\thepage}
\fancyhead[LO]{\leftmark}
\fancyhead[RE]{\rightmark}
\cfoot{Physik 1: Mechanik}
\lfoot{Roberto Cuervo \'Alvarez}
\rfoot{\ddmmyyyydate\today}
\renewcommand{\headrulewidth}{0.4pt}
\renewcommand{\footrulewidth}{0.4pt}


%begin Document
\begin{document}
%Title
\begin{titlepage}
\setlength{\parindent}{0pt}
\setlength{\parskip}{0pt}
\vspace*{\stretch{1}}
\rule{\linewidth}{1pt}
\begin{flushright}
\Huge Semester Arbeit \\[14pt]
\Huge Augmented Reality Android App \\[14pt]
HSR HS2016\\Roberto Cuervo \'Alvarez\\Konrad H�pli
\end{flushright}
\rule{\linewidth}{2pt}
\vspace*{\stretch{2}}
\pagenumbering{gobble}
\end{titlepage}

\newpage
\pagenumbering{arabic}
%Table of contents
\renewcommand*\contentsname{Inhaltsverzeichnis}
\tableofcontents
\newpage
%Table of images
\renewcommand*\listfigurename{Abbildungsverzeichnis}
\listoffigures
\newpage

\section{Introduction}
\subsection{Introduction}
\subsubsection{Introduction}
\end{document}